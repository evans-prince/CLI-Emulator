\documentclass[12pt]{article}
\usepackage[margin=1in]{geometry}
\usepackage{titlesec}
\usepackage{enumitem}
\usepackage{amsmath}
\usepackage{listings}
\usepackage{xcolor}
\usepackage{hyperref}
\usepackage{array}

\titleformat{\section}{\large\bfseries}{\thesection}{1em}{}

\title{\textbf{CLI - Emulator}}
\author{Prince Evans }
\date{}
\begin{document}

\maketitle

\section*{Overview}
\texttt{CLI - Emulator} (Command Line - Emulator) is a C program which reads a SimpleRisc assembly language file and executes its instructions to produce the desired output.

\section*{Features of the Interpreter}
\begin{itemize}[leftmargin=*]
  \item Supports all 21 SimpleRisc instructions including modifiers \textbf{u} and \textbf{h}.
  
  \item The assemble language program can have a \textbf{free form}. An instruction has to be in one
line but there can be any number of spaces or tabs between the fields and before the
instruction. There can also be any number of empty lines between statements.

  \item The program specifies any errors in the assembly file and \textbf{also mentions the line
number} where the error is present.

  \item The assembly file can have any number of labels and a label can either be on the same
line as the instruction it points to, or it can be anywhere after the previous instruction.

  \item The program must contain a \textbf{.main} label from where the program execution starts.
  
  \item The stack pointer, \textbf{r14}, can be accessed as sp and the return address register, \textbf{r15}, can be
accessed as ra.

  \item 16-bit immediates (in either decimal or hex) are supported and can have the form \texttt{21, -53,
0xAB12, 0x a25 or 0x  5     A 4    C.}

  \item It supports a 32-bit memory system. However, only the bottom 4096 entries are allowed
to be used. The interpreter allows only memory addresses that are multiples of 4 between
0 and 4095 to be accessed using the ld and st instructions.

  \item The stack pointer is initially automatically initialized to 0xFFC, i.e. 4092 (the upper end
of the memory) and hence the program need not initialize the stack pointer.

  \item Supports single-line (\textbf{@}) and multi-line (\texttt{/*...*/}) comments.
  
  \item Includes two macros: \texttt{.print} and \texttt{.encode}.
\end{itemize}

\section*{The \texttt{.print} Macro}

Used to print decimal values of registers to console.  
Example:
\begin{lstlisting}[language={},basicstyle=\ttfamily]
.print r0, r2, r11, r9, sp, ra, r6
\end{lstlisting}

\section*{The \texttt{.encode} Macro}
Used to print the 32-bit encoding of a SimpleRisc instruction.  
Example:
\begin{lstlisting}[language={},basicstyle=\ttfamily]
.encode addh r15, sp, 0x135
\end{lstlisting}
This prints the encoding \texttt{0x07fa0135}, without executing the instruction.

\section*{How to Use the Emulator}
\begin{itemize}[leftmargin=*]
  \item Compile \texttt{main.c} (use \texttt{make} if a Makefile is provided).
  \item Run using:
  \begin{itemize}
    \item \texttt{./<executable> <assembly-file>} (Linux)
    \item \texttt{<executable> <assembly-file>} (Windows)
    \item \texttt{./run.sh <assembly-file>} (if Makefile is used)
  \end{itemize}
\end{itemize}


\end{document}
